\documentclass{beamer}
\usepackage{polski}
\usepackage[utf8x]{inputenc}
\usepackage{graphicx}
\usepackage{hyperref}

\usetheme[hideothersubsections]{Berkeley}

\title[Fizyka plazmy]{Energetyka Jądrowa++}
\subtitle{Czyli wszystko co chcielibyście wiedzieć o plazmach gdybyście
wiedzieli, że są one, a są one}
\institute{Politechnika Warszawska}
\author{Dominik Stańczak}
\date{20 grudnia 2016, Wstęp do Fizyki Jądrowej}

\newcommand {\framedgraphic}[1] {
    % \begin{frame}
        \begin{center}
            \includegraphics[width=\textwidth,height=0.8\textheight,keepaspectratio]{#1}
        \end{center}
    % \end{frame}
}

\begin{document}
  \frame{\titlepage}

  \section{Energetyka jądrowa w tę i we wtę}
  \subsection{Energia wiązania}

  \begin{frame}
    \frametitle{Energia wiązania}
    \framedgraphic{img/binding_energy.png}
  \end{frame}


  \begin{frame}
    \frametitle{Ile z was myśli, że...}
    \begin{itemize}[<+->]
      \item W tej analogii coś jest?
      \item W tej analogii jest na tyle czegoś, żeby była możliwa z tego produkcja energii?
      \item W ten sposób możnaby za naszego życia generować energię?
      \item W ten sposób już teraz generowana energia?
      \item Widziało ostatnio coś, co używało tej zasady?
    \end{itemize}
   \end{frame}


  \begin{frame}
    \frametitle{Odpowiedź jest jasna.}
    \framesubtitle{Słońce, no lol.}
    \framedgraphic{img/large_sun}
  \end{frame}
  \begin{frame}[t]{Fuzja jądrowa wodoru!}
    \framedgraphic{img/hh_fusion}
  \end{frame}


  \begin{frame}
    \frametitle{Energia wiązania raz jeszcze}
    \begin{figure}
      \framedgraphic{img/binding_energy_star}
    \end{figure}
  \end{frame}

  \subsection{Prosty model - jakich energii potrzebujemy?}
  \begin{frame}
    \frametitle{Mechanika fuzji jądrowej}
    \framedgraphic{img/energia1.png}
  \end{frame}

  \begin{frame}
    \frametitle{Mechanika (klasyczna) fuzji jądrowej}
    \framedgraphic{img/energia2.png}
  \end{frame}


  \begin{frame}
    \frametitle{Kicha}
    \framesubtitle{Klasyczna}
    \pause
    \framedgraphic{img/kaszanka.jpg}
  \end{frame}

  \begin{frame}[t]{Cytacik}
    We do not argue with the critic who urges that the stars are not hot enough for this process; we tell him to go and find a hotter place. - Arthur Eddington
    \framedgraphic{img/Eddington}
  \end{frame}


  \begin{frame}
    \frametitle{Mechanika (kwantowa, znaczy lepsza) fuzji jądrowej}
    \framedgraphic{img/energia3.png}
  \end{frame}

  \section{Praktyczne sposoby na ogólne koncepty}
  \subsection{Duże ciśnienia, mniejsze temperatury}
  \begin{frame}
    \frametitle{Duże ciśnienia, małe temperatury}
    \framedgraphic{img/ananas}
  \end{frame}

  \begin{frame}
    \frametitle{Jak to jest w Słońcu?}
    \framedgraphic{img/wikipedia_solarcore}
  \end{frame}

  \begin{frame}
    \frametitle{Ile wyciskamy w praktyce?}
    \framedgraphic{img/pressure_record}
  \end{frame}

  \subsection{Duże temperatury, mniejsze ciśnienia}
  \begin{frame}
    \frametitle{Duże temperatury, mniejsze ciśnienia}
    \framedgraphic{img/campfire.png}
  \end{frame}

  \begin{frame}
    \frametitle{Do ilu dochodzimy w praktyce?}
    \begin{columns}[c]
      \column{.5\textwidth}
      \framedgraphic{img/wiki_T}
      \column{.5\textwidth}
      \pause

      \framedgraphic{img/facebook_like}

      \pause

      \framedgraphic{img/git_logo}
    \end{columns}
  \end{frame}

  \section{Stany materii}
  \subsection{Odmienne stany skupienia}
  \begin{frame}
    \frametitle{Podgrzewanie materii}
    \framesubtitle{Stan stały}
    \framedgraphic{img/solid}
  \end{frame}

  \begin{frame}
    \frametitle{Podgrzewanie materii}
    \framesubtitle{Stan ciekły}
    \framedgraphic{img/liquid}
  \end{frame}


  \begin{frame}
    \frametitle{Podgrzewanie materii}
    \framesubtitle{Stan gazowy}
    \framedgraphic{img/gas}
  \end{frame}

  \begin{frame}
    \frametitle{Podgrzewanie materii}
    \framesubtitle{Stan gazowy}
    \framedgraphic{img/gas2}
  \end{frame}

  \begin{frame}
    \frametitle{Podgrzewanie materii}
    \framesubtitle{Kompletny wykres przejść fazowych}
    \framedgraphic{img/phasetransition}
  \end{frame}

  \begin{frame}
    \frametitle{Podgrzewanie materii}
    \framesubtitle{Gdzie jeszcze mamy drogę ujścia dla energii?}
    \framedgraphic{img/plasma_zoom}
  \end{frame}

  \begin{frame}
    \frametitle{Podgrzewanie materii}
    \framesubtitle{Gdzie jeszcze mamy drogę ujścia dla energii?}
    \framedgraphic{img/plasma_ionise}
  \end{frame}

  \begin{frame}[t]{Oddziaływania}
    % TODO
    \framedgraphic{img/plasma_ionise_novectors.png}
  \end{frame}

  \begin{frame}[t]{Oddziaływania}
    \framesubtitle{Każdy z każdym!}
      % TODO
    \framedgraphic{img/plasma_ionise_vectors.png}
  \end{frame}



  \begin{frame}
    \frametitle{Zoom out na skalę kosmiczną}
    99\% obserwowalnej materii we wszechświecie to plazma.
    \pause
    \framedgraphic{img/starrynight}
    \pause
    To nie plazma jest taka wyjątkowa, to my, jako nie(plazma), jesteśmy specjalni!
  \end{frame}

\subsection{Przykłady występowania}

\begin{frame}
  \frametitle{Zjonizowany gaz...}
  \framesubtitle{Może warto poszukać w powietrzu, c'nie?}
  \framedgraphic{img/lightning.png}
\end{frame}

\begin{frame}[t]{Zorze polarne, magnetosfera}
  \framedgraphic{img/AuroraNorway.jpg}
\end{frame}

\begin{frame}[t]{Ogień}
  \framedgraphic{img/campfire.png}
\end{frame}

\begin{frame}[t]{Świetlówki}
  \framedgraphic{img/fluorescent}
\end{frame}

\section{Jak ugryźć plazmę?}
\begin{frame}[t]{Jak ugryźć plazmę?}
  \framesubtitle{Z czym to się je?}
  \pause
    \begin{alertblock}{UWAGA!}
    PROSZĘ NIE GRYŹĆ PLAZMY.

    GRYZIENIE PLAZMY MOŻE SKUTKOWAĆ ŚMIERCIĄ.

    PRZED UGRYZIENIEM SKONSULTUJ SIĘ Z FIZYKIEM MEDYCZNYM LUB FARMACEUTĄ.
  \end{alertblock}
\end{frame}


\subsection{Opis teoretyczny}

\begin{frame}[t]{Elektrodynamika}
  \begin{itemize}[<+->]
    \item Ruch materii naładowanej - siła Lorentza jako podstawowa
    \item Równania Maxwella - ewolucja pola elektromagnetycznego
    \item Relatywistyka też potrafi być konieczna - całkiem wysokie energie, akceleratory
  \end{itemize}
\end{frame}

\begin{frame}[t]{Hydrodynamika}
  \begin{itemize}[<+->]
    \item Kolektywne zachowania wielu cząstek
    \item Równania Naviera-Stokesa
      \begin{equation}
      \frac{\partial(\rho e)}{\partial t} + \overrightarrow{\nabla}\cdot((\rho e + p)\overrightarrow{u}) = \overrightarrow{\nabla}\cdot(\overline{\overline{\tau}}\cdot\overrightarrow{u}) + \rho\overrightarrow{f}\overrightarrow{u} + \overrightarrow{\nabla}\cdot(\overrightarrow{\dot{q}})+r \end{equation}

    \item ... z wpiętymi równaniami Maxwella
    \item \textbf{magnetohydrodynamika}
  \end{itemize}

\end{frame}

\begin{frame}[t]{Mechanika klasyczna i dynamika nieliniowa}
  \begin{itemize}[<+->]
    \item Często układy prawie zachowawcze (opór!)
    \item Chaos deterministyczny
    \item Niestabilności
    \item Modele na \em{cząstkach} używane do badania \em{turbulencji}
    \item Heisenberg na łożu śmierci podobno rzekł:
    \item "Gdy spotkam Boga, zamierzam mu zadać dwa pytania.
    \item Dlaczego relatywistyka?
    \item Dlaczego turbulencja?
    \item Wierzę, że będzie miał odpowiedź na pierwsze."
  \end{itemize}
\end{frame}

\begin{frame}[t]{Mechanika kwantowa}
  ... no w sumie to rzadko.
  \framedgraphic{img/google_plasma.png}
  \pause
  \framedgraphic{img/google_qplasma.png}
\end{frame}

\begin{frame}[t]{Mechanika statystyczna}
  \begin{itemize}[<+->]
    \item Układy daleko od równowagi - fizyka nierównowagowa
    \item Rzędu $10^{23}$ cząstek
    \item Złożoność oddziaływań $n^2$
    \item Mierzalne jedynie wartości średnie i ich odchylenia
    \item Równanie Vlasova-Maxwella
    \begin{align}
      \frac{\partial f_e}{\partial t} + \mathbf {v}_e\cdot\nabla f_e &- e\left(\mathbf {E}+\frac{\mathbf {v_e}}{c}\times\mathbf {B}\right)\cdot\frac{\partial f_e}{\partial\mathbf {p}} = 0 \\
      \frac{\partial f_i}{\partial t} + \mathbf {v}_i\cdot\nabla f_i &+ Z_i e\left(\mathbf {E}+\frac{\mathbf {v_i}}{c}\times\mathbf {B}\right)\cdot\frac{\partial f_i}{\partial\mathbf {p}} = 0 \\
      \nabla\times\mathbf {B} &=\frac{4\pi\mathbf {j}}{c}+\frac{1}{c}\frac{\partial\mathbf {E}}{\partial t} \\
      \nabla\times\mathbf {E} &=-\frac{1}{c}\frac{\partial\mathbf {B}}{\partial t} \\
      \nabla\cdot\mathbf {E}  &=4\pi\rho \\
      \nabla\cdot\mathbf {B}  &=0 \\
    \end{align}
  \end{itemize}
\end{frame}

\subsection{Badania doświadczalne}
\begin{frame}[t]{Badania doświadczalne}
  \framesubtitle{RGDX}
  \href{http://scied-web.pppl.gov/rgdx/}{\textbf{Remote Control Glow Discharge Experiment}}
  na stronie Princeton Plasma Physics Laboratory

  http://scied-web.pppl.gov/rgdx/
\end{frame}


\section{Zastosowania}
\begin{frame}[t]{Po... co nam to?}
  \framedgraphic{img/nachujmilas.jpg}
\end{frame}

\subsection{Zastosowania przemysłowe}
\begin{frame}[t]{Obróbka mechaniczna}
  \framedgraphic{img/plasmacutter}
\end{frame}
\begin{frame}[t]{Obróbka powierzchniowa}
  \framedgraphic{img/Plasma_Spraying_Process}
\end{frame}

\begin{frame}[t]{Plasma CVD}
  \framedgraphic{img/pcvd}
\end{frame}

\begin{frame}[t]{Plasma etching}
  \framedgraphic{img/plasmaetching.png}
\end{frame}

\subsection{Zastosowania nieco bardziej}
\begin{frame}[t]{Medycyna plazmowa}
  \begin{columns}[c]
    \column{.5\textwidth}
    \framedgraphic{img/coldplasmamedicine}
    \column{.5\textwidth}
    \begin{itemize}[<+->]
      \item Plazma składa się z naładowanych, aktywnych chemicznie cząstek, jest
      źródłem fotonów ultrafioletowych
      \item Strumienie plazmy pod atmosferycznym ciśnieniem wykorzystywane do
      usuwania bakterii oraz infekcji podskórnych
      \item Zdaje się przyspieszać gojenie ran, nie wiemy dlaczego
    \end{itemize}
  \end{columns}
\end{frame}

\begin{frame}[t]{Silniki kosmiczne}
  \framedgraphic{img/Xenon_hall_thruster}
\end{frame}

\begin{frame}[t]{Akceleratory plazmowe}
  \framedgraphic{img/plasmaaccelerator}
\end{frame}

\subsection{Fuzja termojądrowa}
\begin{frame}[t]{Nie tylko fikcja, ale ku pamięci...}
  \begin{columns}[c]
    \column{.5\textwidth}
    \framedgraphic{img/spiderman2_fusion}
    \framedgraphic{img/mrfusion1}
    \column{.5\textwidth}
    \framedgraphic{img/ironman_tokamak}
    \framedgraphic{img/interstellar}
  \end{columns}
\end{frame}

\begin{frame}[t]{Ivy Mike}
  \framedgraphic{img/IvyMike}
\end{frame}

\begin{frame}[t]{Zwierciadła magnetyczne}
  \framedgraphic{img/magmirror}
\end{frame}

\begin{frame}[t]{Reaktory toroidalne}
  \framesubtitle{Tokamaki}
  \framedgraphic{img/tokamak_scheme}
\end{frame}

\begin{frame}[t]{COMPASS}
  \framesubtitle{Tokamaki}
  \framedgraphic{img/compass}
\end{frame}

\begin{frame}[t]{ASDEX}
  \framesubtitle{Tokamaki}
  \framedgraphic{img/asdex}
\end{frame}

\begin{frame}[t]{JET}
  \framesubtitle{Tokamaki}
  \framedgraphic{img/jet}
\end{frame}

\begin{frame}[t]{ITER}
  \framesubtitle{Tokamaki}
  \framedgraphic{img/ITERsketch}
\end{frame}

\begin{frame}[t]{ITER}
  \framesubtitle{Tokamaki}
  \framedgraphic{img/IterPit}
\end{frame}

\begin{frame}[t]{Stellarator}
  \framedgraphic{img/Wendelstein}
\end{frame}

\begin{frame}[t]{Wendelstein 7-X}
  \framesubtitle{Stellarator}
  \framedgraphic{img/wendelstein_magsurfaces}
\end{frame}

\begin{frame}[t]{Wendelstein 7-X}
  \framesubtitle{Stellarator}
  \framedgraphic{img/w7x-diag}
\end{frame}

\begin{frame}[t]{Tokamaki sferyczne}
  \framedgraphic{img/spherical_tokamak}
\end{frame}

\begin{frame}[t]{Inertial Confinement Fusion}
  \framesubtitle{National Ignition Facility, USA}
  A może po prostu walnąć w kapsułkę z paliwem ogromnym laserem?
  \pause
  \framedgraphic{img/NIF_fuel}
\end{frame}

\begin{frame}[t]{I tyle w temacie!}
  Wesołych świąt i samych gwiazdek z nieba!
\end{frame}

\end{document}
