\documentclass{beamer}
\usepackage{polski}
\usepackage[utf8x]{inputenc}
% \usepackage[unicode]{hyperref}
% \usepackage{amssymb}
% \usepackage{xifthen}
% \usepackage[fleqn]{amsmath}
\usepackage{graphicx}
\usetheme[hideothersubsections]{Berkeley}

\title[Fizyka plazmy]{Energetyka Jądrowa++}
\subtitle{Czyli wszystko co chcielibyście wiedzieć o plazmach gdybyście
wiedzieli, że są one, a są one}
\institute{Politechnika Warszawska}
\author{Dominik Stańczak}
\date{20 grudnia 2016, Wstęp do Fizyki Jądrowej}

\newcommand {\framedgraphic}[1] {
    % \begin{frame}
        \begin{center}
            \includegraphics[width=\textwidth,height=0.8\textheight,keepaspectratio]{#1}
        \end{center}
    % \end{frame}
}

\begin{document}
  \frame{\titlepage}

  \section{Energetyka jądrowa w tę i we wtę}
  \subsection{Energia wiązania}

  \begin{frame}
    \frametitle{Energia wiązania}
    \framedgraphic{img/binding_energy.png}
  \end{frame}


  \begin{frame}
    \frametitle{Ile z was myśli, że...}
    \begin{itemize}[<+->]
      \item W tej analogii coś jest?
      \item W tej analogii jest na tyle czegoś, żeby była możliwa z tego produkcja energii?
      \item W ten sposób możnaby za naszego życia generować energię?
      \item W ten sposób już teraz generowana energia?
      \item Widziało ostatnio coś, co używało tej zasady?
    \end{itemize}
   \end{frame}


  \begin{frame}
    \frametitle{Odpowiedź jest jasna.}
    \framesubtitle{Słońce, no lol.}
    \framedgraphic{img/large_sun}
  \end{frame}
  \begin{frame}[t]{Fuzja jądrowa wodoru!}
    \framedgraphic{img/hh_fusion}
  \end{frame}


  \begin{frame}
    \frametitle{Energia wiązania raz jeszcze}
    \begin{figure}
      \framedgraphic{img/binding_energy}
    \end{figure}
  \end{frame}

  \subsection{Prosty model - jakich energii potrzebujemy?}
  \begin{frame}
    \frametitle{Mechanika fuzji jądrowej}
    % wykres
  \end{frame}

  \begin{frame}
    \frametitle{Mechanika (klasyczna) fuzji jądrowej}
    % wykres
  \end{frame}


  \begin{frame}
    \frametitle{Kicha}
    \framesubtitle{Klasyczna}
    blah
    \pause
    % TODO: zdjęcie kaszanki
  \end{frame}


  \begin{frame}
    \frametitle{Mechanika (kwantowa, znaczy lepsza) fuzji jądrowej}
    % Tunelowanie kwantowe!
  \end{frame}

  \section{Praktyczne sposoby na ogólne koncepty}
  \subsection{Duże ciśnienia, mniejsze temperatury}
  \begin{frame}
    \frametitle{Duże ciśnienia, małe temperatury}
    % prasa z arbuzem
  \end{frame}

  \begin{frame}
    \frametitle{Jak to jest w Słońcu?}
    % solar core
  \end{frame}

  \begin{frame}
    \frametitle{Ile wyciskamy w praktyce?}
    % record high
    \pause
    % sadface
  \end{frame}

  \subsection{Duże temperatury, mniejsze ciśnienia}
  \begin{frame}
    \frametitle{Duże temperatury, mniejsze ciśnienia}
    % ognisko
  \end{frame}

  \begin{frame}
    \frametitle{Do ilu dochodzimy w praktyce?}
    % tabelka wiki
    \pause
    % okejka
    \pause
    % logo git
  \end{frame}

  \section{Stany materii}
  \subsection{Odmienne stany skupienia}
  \begin{frame}
    \frametitle{Podgrzewanie materii}
    \framesubtitle{Stan stały}
    % obrazek
  \end{frame}

  \begin{frame}
    \frametitle{Podgrzewanie materii}
    \framesubtitle{Stan ciekły}
    % obrazek
  \end{frame}


  \begin{frame}
    \frametitle{Podgrzewanie materii}
    \framesubtitle{Stan gazowy}
    % obrazek
    \pause
    % interakcja!!!!
  \end{frame}


  \begin{frame}
    \frametitle{Podgrzewanie materii}
    \framesubtitle{Gdzie jeszcze mamy drogę ujścia dla energii?}
    % obrazek
  \end{frame}


  \begin{frame}
    \frametitle{Podgrzewanie materii}
    \framesubtitle{Kompletny wykres przejść fazowych}
    % obrazek
  \end{frame}


  \begin{frame}
    \frametitle{Podgrzewanie materii}
    \framesubtitle{Plazma}
    % obrazek
    \pause
    % oddziaływania
  \end{frame}


  \begin{frame}
    \frametitle{Zoom out na skalę kosmiczną}
    99\% obserwowalnej materii we wszechświecie to plazma.
    \pause
    % obrazek - gwiazdy
    \pause
    To nie plazma jest taka wyjątkowa, to my, jako ~plazma, jesteśmy specjalni!
    % obrazek - special snowflake
  \end{frame}

\subsection{Przykłady występowania}

\begin{frame}
  \frametitle{Zjonizowany gaz...}
  \framesubtitle{Może warto poszukać w powietrzu, c'nie?}
  % obrazek błyskawicy
\end{frame}

% TODO: dodac przykłady występowania plazmy ziemskiej
\begin{frame}[t]{Świetlówki}[s]{światło dają, no}
  body
\end{frame}
%--- Next Frame ---%

\subsection{Zakres parametrów}
\begin{frame}[t]{Rozmiary}
  body
\end{frame}
%--- Next Frame ---%

\begin{frame}[t]{Czas "życia"}
  body
\end{frame}
%--- Next Frame ---%

\begin{frame}[t]{Koncentracja cząstek}
  body
\end{frame}
%--- Next Frame ---%

\begin{frame}[t]{Siła pola magnetycznego}
  body
\end{frame}
%--- Next Frame ---%

\begin{frame}[t]{Magnetary}
  body
\end{frame}
%--- Next Frame ---%

\section{Jak ugryźć plazmę?}
\begin{frame}[t]{UWAGA}
  PROSZĘ NIE GRYŹĆ PLAZMY.

  GRYZIENIE PLAZMY MOŻE SKUTKOWAĆ ŚMIERCIĄ.

  PRZED UGRYZIENIEM SKONSULTUJ SIĘ Z FIZYKIEM MEDYCZNYM LUB FARMACEUTĄ.
\end{frame}
%--- Next Frame ---%
\subsection{Opis teoretyczny}

\begin{frame}[t]{Fizyka statystyczna}
  body
\end{frame}
%--- Next Frame ---%


\begin{frame}[t]{Elektrodynamika}
  body
\end{frame}
%--- Next Frame ---%

\begin{frame}[t]{Hydrodynamika}
  body
\end{frame}
%--- Next Frame ---%

\begin{frame}[t]{Mechanika}
  body
\end{frame}
%--- Next Frame ---%

\subsection{Symulacje komputerowe}

\subsection{Badania doświadczalne}


\section{Zastosowania}
\begin{frame}[t]{Po... co nam to?}
  % po chuj mi las
\end{frame}
%--- Next Frame ---%

\subsection{Zastosowania przemysłowe}
\begin{frame}[t]{title}
  body
\end{frame}
\subsection{Zastosowania "kreatywne"}

\subsection{Fuzja termojądrowa}
\begin{frame}[t]{Tokamaki}
  body
\end{frame}
%--- Next Frame ---%


%--- Next Frame ---%

\section{Co tu jest do roboty?}
\begin{frame}[t]{title}
  body
\end{frame}
%--- Next Frame ---%
\end{document}
